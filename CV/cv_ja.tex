%%%%%%%%%%%%%%%%%%%%%%%%%%%%%%%%%%%%%%%%%
% Medium Length Professional CV
% LaTeX Template
% Version 2.0 (8/5/13)
%
% This template has been downloaded from:
% http://www.LaTeXTemplates.com
%
% Original author:
% Trey Hunner (http://www.treyhunner.com/)
%
% Important note:
% This template requires the resume.cls file to be in the same directory as the
% .tex file. The resume.cls file provides the resume style used for structuring the
% document.
%
%%%%%%%%%%%%%%%%%%%%%%%%%%%%%%%%%%%%%%%%%

%----------------------------------------------------------------------------------------
%	PACKAGES AND OTHER DOCUMENT CONFIGURATIONS
%----------------------------------------------------------------------------------------

\documentclass{resume} % Use the custom resume.cls style

\include{macro}	% マクロを読み込む

\usepackage[left=0.75in,top=0.6in,right=0.75in,bottom=0.6in]{geometry} % Document margins
\usepackage[dvipdfmx]{hyperref}
\newcommand{\tab}[1]{\hspace{.2667\textwidth}\rlap{#1}}
\newcommand{\itab}[1]{\hspace{0em}\rlap{#1}}
\name{村松 直哉}	% Your name
% \address{\href{http://www.u.tsukuba.ac.jp/~s1411453/}{http://www.u.tsukuba.ac.jp/~s1411453/}}
\address{sh.mn.nat@gmail.com}	% Your phone number and email
\address{}	% Your address
% \address{123 Pleasant Lane \\ City, State 12345} % Your secondary address (optional)


\begin{document}

%----------------------------------------------------------------------------------------
%	EDUCATION SECTION
%----------------------------------------------------------------------------------------

\begin{rSection}{学歴}

    {\bf 筑波大学} \hfill {2018年4月 -- 現在}
    \\ 修士(情報学)
    \\ 図書館情報メディア研究科 図書館情報メディア専攻
    \\ 指導教員: 佐藤 哲司 教授
    \\ 副指導教員: 于 海涛 助教

    {\bf 筑波大学} \hfill {2016年4月 -- 2018年3月}
    \\ 学士(図書館情報学)
    \\ 情報学群 知識情報・図書館学類
    \\ 指導教員: 落合 陽一 准教授
    % \\ GPA: \todo{xxx / 4.03}

    {\bf 長野工業高等専門学校} \hfill {2011年4月 -- 2016年3月}
    \\ 準学士
    \\ 電気電子工学科
    \\ 指導教員: 宮崎 敬 教授
    % \\ GPA: \todo{xxx / 4.03}

\end{rSection}

%----------------------------------------------------------------------------------------
%	WORK EXPERIENCE SECTION
%----------------------------------------------------------------------------------------

\begin{rSection}{研究}

    \begin{rSubsection}{筑波大学}{2018年4月 -- 現在}{修士論文}{}
    % New type of deep learning
    \item スパイキングニューラルネットワーク(SNN)と従来のニューラルネットワークを組み合わせたハイブリッドネットワークの提案。次世代ニューラルネットワークであるSNNの性能向上に貢献。
    \end{rSubsection}

    \begin{rSubsection}{ケープタウン大学}{2020年7月}{リサーチインターン}{}
    % New type of deep learning
    \item 軌道最適化を用いたマルチカメラにより記録したチーターの走行解析(指導教員: \href{https://scholar.google.co.za/citations?user=RxMigV4AAAAJ&view_op=list_works&sortby=pubdate}{Dr. Amir Patel})。
    \end{rSubsection}

    \begin{rSubsection}{筑波大学}{2016年4月 -- 2018年3月}{卒業研究}{}
    % ECサイトのレビュー
    \item ECサイトのレビュー傾向の時間変化を分析。ユーザーの趣向が時間的に変化するパターンを明らかにした。
    % Sonoliards
    \item レイトラッキングを用いた指向性スピーカーの音響制御に関す研究。障害物を回避して、特定の人物に音を届けるアルゴリズムを開発。
    % DeepHolo
    \item DeepHoloという3次元認識アルゴリズムを研究。深層学習とCGHを組み合わせることで、モデルの大幅な軽量化に成功。
    % DeepWear
    \item DCGANを用いて衣装をデザインし、実際の服を制作する研究。デザイナーと協力し、実際に服を作り、NIPSのワークショップで発表。
    \end{rSubsection}

    \begin{rSubsection}{長野工業高等専門学校}{2015年4月 -- 2016年3月}{卒業研究}{}
    \item 小型センサネットワークを用いた屋内位置情報システムの開発と、電波強度による位置情報精度向上フィルタの提案。
    \end{rSubsection}

\end{rSection}

%----------------------------------------------------------------------------------------
%	TECHNICAL STRENGTHS SECTION
%----------------------------------------------------------------------------------------

\begin{rSection}{技術スキル}

    \begin{tabular}{ @{} >{\bfseries}l @{\hspace{6ex}} l }
    プログラミング言語   &   Python, C, C++, Verilog, Shell Script, \\
    & Ruby, JavaScript, SQL \\
    機械学習 & Tensorflow, Keras, PyTorch, Scikit-learn, Chainer \\
    ソフトウェア    &   Git, Docker, PyBullet, Processing, Autodesk Fusion360 \\
    OS          &   MacOS, Ubuntu, Windows, FreeNAS, CentOS \\
    ハードウェア    &   Arduino, Mbed, PhantomX AX Metal Hexapod \\
    \end{tabular}

\end{rSection}

%----------------------------------------------------------------------------------------
%	EXAMPLE SECTION
%----------------------------------------------------------------------------------------

\begin{rSection}{Publications} \itemsep 4pt

    % \begin{rSubsection}{Preprints}{}{}{}{}
    %     % arxivとか
    % \end{rSubsection}

    % \begin{rSubsection}{JOURNAL ARTICLES}{}{}{}{}
    % \end{rSubsection}

    \begin{rSubsection}{査読あり国際会議 (登壇)}{}{}{}{}
        \item Daniel Joska, Liam Clark, \textbf{Naoya Muramatsu}, Ricky Jericevich, Fred Nicolls, Alexander Mathis, Mackenzie Mathis, Amir Patel. 2021. AcinoNet: 3D Markerless Motion Tracking of Cheetahs in the Wild. In \textit{International Conference on Robotics and Automation} (ICRA 2021). IEEE, Xi' an, China.

        \item Chun Wei Ooi, \textbf{Naoya Muramatsu}, and Yoichi Ochiai. 2018. Eholo glass: Electroholography glass. A lensless approach to holographic augmented reality near-eye display. In \textit{SIGGRAPH Asia 2018 Technical Briefs} (SA ’18), December 4–7, 2018, Tokyo, Japan. ACM, New York, NY, USA, 4 pages. DOI: \url{https://doi.org/10.1145/3283254.3283288}

        \item Natsumi Kato*, Hiroyuki Osone*, Daitetsu Sato, \textbf{Naoya Muramatsu}, and Yoichi Ochiai. 2018. DeepWear: a Case Study of Collaborative Design between Human and Artificial Intelligence. In \textit{Proceedings of the Twelfth International Conference on Tangible, Embedded, and Embodied Interaction} (TEI ’18). ACM, New York, NY, USA, 529-536. DOI: \url{https://doi.org/10.1145/3173225.3173302} (* Joint first authorship.)

        \item \textbf{Naoya Muramatsu}, Ooi Chun Wei, Takashi Miyazaki. 2017. Development of High Performance Filter for Indoor Positioning System. In \textit{The 5th IIAE International Conference on Intelligent Systems and Image Processing 2017}(ICISIP 2017).
    \end{rSubsection}

    \begin{rSubsection}{査読あり国際学会 (ポスター・ワークショップ)}{}{}{}{}
        \item Natsumi Kato, Hiroyuki Osone, Daitetsu Sato, \textbf{Naoya Muramatsu}, and Yoichi Ochiai. 2017. Crowd Sourcing Clothes Design Directed by Adversarial Neural Networks. In \textit{NIPS 2017 Workshop} (NIPS ’17).

        \item \textbf{Naoya Muramatsu}, Kazuki Ohshima, Ryota Kawamura, Ooi Chun Wei, Yuta Sato, and Yoichi Ochiai. 2017. Sonoliards: Rendering Audible Sound Spots by Reflecting the Ultrasound Beams. In \textit{Adjunct Publication of the 30th Annual ACM Symposium on User Interface Software and Technology} (UIST ’17). ACM, New York, NY, USA, 57-59. DOI: \url{https://doi.org/10.1145/3131785.3131807}

        \item \textbf{Naoya Muramatsu}, Chun Wei Ooi, Yuta Itoh, and Yoichi Ochiai. 2017. DeepHolo: Recognizing 3D Objects using a Binary-weighted Computer-Generated Hologram. In \textit{SIGGRAPH Asia 2017 Posters} (SA 2017), November 27--30, 2017, Bangkok, Thailand. ACM, New York, NY, USA, 2 pages. DOI: \url{https://doi.org/10.1145/3145690.3145725}

        \item Mose Sakashita, Yuta Sato, Ayaka Ebisu, Keisuke Kawahara, Satoshi Hashizume, \textbf{Naoya Muramatsu}, Yoichi Ochiai. 2017. Haptic Marionette: Wrist Control Technology Combined with Electrical Muscle Stimulation and Hanger Reflex. In \textit{SIGGRAPH Asia 2017 Posters} (SA 2017). ACM, New York, NY, USA, Article 33, 2 pages. DOI: \url{https://doi.org/10.1145/3145690.3145743}
    \end{rSubsection}

    % \begin{rSubsection}{国際学会 (NOT REFEREED)}{}{}{}{}
    % \end{rSubsection}

    % \begin{rSubsection}{DOMESTIC CONFERENCES (査読あり)}{}{}{}{}
    % \end{rSubsection}

    \begin{rSubsection}{査読なし国内学会}{}{}{}{}
        \item \textbf{Naoya Muramatsu}, Hai-Tao Yu. 2021. "Combining Spiking Neural Network and Artificial Neural Network for Enhanced Image Classification", 第13回データ工学と情報マネジメントに関するフォーラム (DEIM 2021).
        \item \textbf{村松直哉}, 佐藤哲司, 伏見卓恭. 2017. "レビュー観点の推移パターンに基づく商品属性の抽出手法", 第9回データ工学と情報マネジメントに関するフォーラム (DEIM2017)
    \end{rSubsection}




\end{rSection}

%----------------------------------------------------------------------------------------
\begin{rSection}{職歴}

    \begin{rSubsection}{\href{https://www.landscape.co.jp/}{株式会社ランドスケイプ}}{2020年1月 -- 現在}{社外CTO}{}
    \item 機械学習プロジェクトの推進と社内教育に従事
    \item 開発した主なシステム: 名刺用OCR、推薦システム、PDF解析プログラム、自然言語解析プログラムなど多数
    \end{rSubsection}

    \begin{rSubsection}{\href{https://www.mitou.org/}{2018年度未踏IT人材発掘・育成事業}}{2018年6月 -- 2019年3月}{クリエーター}{}
    \item 強化学習を用いた6足歩行ロボットの開発(\href{https://www.ipa.go.jp/jinzai/mitou/2018/gaiyou_tn-1}{プロジェクト概要})
    \item 「スーパークリエーター」認定
    \item 採択金額: 2,304,000円 / 9ヶ月
    \end{rSubsection}

    \begin{rSubsection}{\href{http://pixiedusttech.com/}{ピクシーダストテクノロジーズ株式会社}}{2017年8月 -- 2019年4月}{エンジニア}{}
    \item 空間開発事業\href{https://pixiedusttech.com/kotowari/}{KOTOWARI}の立ち上げ及び、技術選定、開発
    \item 3次元データ収集・処理プログラムの開発
    \end{rSubsection}

    \begin{rSubsection}{\href{https://www.fixstars.com/en/}{Fixstars Corporation}}{2016年8月 -- 2016年12月}{エンジニアインターン}{}
    \item 自動運転用画像認識アルゴリズムの研究開発
    \end{rSubsection}

    \begin{rSubsection}{\href{https://www.fixstars.com/en/}{Fixstars Corporation}}{2014年8月 -- 2014年9月}{エンジニアインターン}{}
    \item 車載マイコンのプログラム最適化
    \end{rSubsection}

\end{rSection}

%----------------------------------------------------------------------------------------
% \begin{rSection}{Fellowships}
%     \begin{tabular}{ @{} >{\bfseries}l @{\hspace{4ex}} l }
%     \end{tabular}
% \end{rSection}

%----------------------------------------------------------------------------------------
% \begin{rSection}{Research Grants}
%     \begin{tabular}{ @{} >{\bfseries}l @{\hspace{4ex}} l }
%     \end{tabular}
% \end{rSection}

%----------------------------------------------------------------------------------------
\begin{rSection}{Awards}
    \begin{tabular}{ @{} >{\bfseries}l @{\hspace{4ex}} l }
    2018    &   2018年度未踏IT人材発掘・育成事業 \textbf{スーパークリエーター}認定  \\
    2018    &   筑波大学 \textbf{学生表彰(学長表彰)}  \\
    2017    &   DEIM 2017 \textbf{学生プレゼンテーション賞}.   \\
    2015    &   ロボカップジュニア2015 北信越ブロック \textbf{3位入賞} \\
    \end{tabular}
\end{rSection}


%----------------------------------------------------------------------------------------
\begin{rSection}{Links}

    {\bf プロフィール} \hfill {\url{https://denden047.github.io/}} \\
    {\bf GitHub} \hfill {\url{https://github.com/DenDen047}} \\
    {\bf Google Scholar} \hfill {\url{https://scholar.google.com/citations?user=H7NhyxQAAAAJ}} \\

\end{rSection}

\end{document}
