%%%%%%%%%%%%%%%%%%%%%%%%%%%%%%%%%%%%%%%%%
% Medium Length Professional CV
% LaTeX Template
% Version 2.0 (8/5/13)
%
% This template has been downloaded from:
% http://www.LaTeXTemplates.com
%
% Original author:
% Trey Hunner (http://www.treyhunner.com/)
%
% Important note:
% This template requires the resume.cls file to be in the same directory as the
% .tex file. The resume.cls file provides the resume style used for structuring the
% document.
%
%%%%%%%%%%%%%%%%%%%%%%%%%%%%%%%%%%%%%%%%%

%----------------------------------------------------------------------------------------
%	PACKAGES AND OTHER DOCUMENT CONFIGURATIONS
%----------------------------------------------------------------------------------------

\documentclass{resume} % Use the custom resume.cls style

% コメント埋め込み用マクロ
\usepackage{color}
\usepackage{url}
\newcommand{\todo}[1]{\textsf{\textcolor{red}{\textbf{TODO:} \textit{#1}}}}


% データセット名マクロ
\usepackage{xspace}

% 省略形
\newcommand{\etal}{\textit{et al}.}
\newcommand{\ie}{\textit{i}.\textit{e}.}
\newcommand{\eg}{\textit{e}.\textit{g}.}
	% マクロを読み込む

\usepackage[left=0.75in,top=0.6in,right=0.75in,bottom=0.6in]{geometry} % Document margins
\usepackage[dvipdfmx]{hyperref}
\usepackage{amsmath}

\newcommand{\tab}[1]{\hspace{.2667\textwidth}\rlap{#1}}
\newcommand{\itab}[1]{\hspace{0em}\rlap{#1}}
\name{Naoya Muramatsu}	% Your name
\address{mrmnao001@myuct.ac.za / sh.mn.nat@gmail.com}	% Your phone number and email
% \address{Yashio, Saitama Pref., Japan}	% Your address
% \address{123 Pleasant Lane \\ City, State 12345} % Your secondary address (optional)


\begin{document}

%----------------------------------------------------------------------------------------
%	EDUCATION SECTION
%----------------------------------------------------------------------------------------

\begin{rSection}{Education}

    \begin{rSubsection}{University of Cape Town}{April 2021 -- Present}{PhD student in Electrical Engineering}{}
    % New type of deep learning
    \item PhD Proposal title: ``{\it On the Neuromechanics of the Cheetah}''
    \item Adviser: \href{https://scholar.google.co.za/citations?user=RxMigV4AAAAJ&view_op=list_works&sortby=pubdate}{Amir Patel}
    \end{rSubsection}

    \begin{rSubsection}{University of Tsukuba}{April 2018 -- March 2021}{MSc in Information Science}{}
    % New type of deep learning
    \item Dissertation title: ``{\it SNN Meets ANN: Combining Spiking Neural Network (SNN) and Artificial Neural Network (ANN) for Image Classification}''
    \item Adviser: \href{http://www.slis.tsukuba.ac.jp/~satoh.tetsuji.gf/index.html}{Tetsuji Satoh}
    \item Subadviser: \href{https://scholar.google.co.jp/citations?user=88b2NRsAAAAJ&hl=ja}{Hai-Tao Yu}
    \end{rSubsection}

    \begin{rSubsection}{University of Tsukuba}{April 2016 -- March 2018}{BSc in Library and Information Science}{}
    \item Dissertation title: ``{\it Deep Learning in Reciprocal Lattice Space}''
    \item Adviser: \href{https://scholar.google.co.jp/citations?user=obhH0jkAAAAJ&hl=en}{Yoichi Ochiai}
    \end{rSubsection}

    \begin{rSubsection}{National Institute of Technology, Nagano College}{April 2015 -- March 2016}{Foundation Degree}{}
    \item Dissertation title: ``{\it Indoor location acquisition using a power-saving wireless network}''(省電力無線ネットワークを用いた屋内位置情報取得)
    \item Adviser: \href{https://researchmap.jp/read0178034?lang=en}{Takashi Miyazaki}
    \end{rSubsection}

\end{rSection}

\begin{rSection}{Industrial Experience}

    \begin{rSubsection}{\href{https://info.gbiz.go.jp/hojin/ichiran?hojinBango=2430001082637}{Good Answers, Inc.}}{February 2021 -- Present}{Fellow}{}
    \item Technical consultant and developing the motor control algorithm for electric scooters.
    \end{rSubsection}

    \begin{rSubsection}{\href{https://www.landscape.co.jp/}{uSonar Co.,Ltd. (Landscape Co.,Ltd.)}}{January 2020 -- Present}{Outside CTO}{}
    \item Developing various systems with machine learning techniques, such as optical character recognition and image recognition.
    \end{rSubsection}

    \begin{rSubsection}{\href{https://www.mitou.org/}{Information-technology Promotion Agency, Japan. Exploratory Software Project (MITOU)}}{June 2018 -- March 2019}{Creator}{}
    \item Developed the robot control system, walking even with broken legs using hierarchy Q-learning.
    \end{rSubsection}

    \begin{rSubsection}{\href{http://pixiedusttech.com/}{Pixie Dust Technologies, Inc.}}{August 2017 -- April 2019}{Software Engineer}{}
    \item Acquired a new big project and worked on the development of management systems and web applications.
    \end{rSubsection}

    \begin{rSubsection}{\href{https://www.fixstars.com/en/}{Fixstars Corporation}}{August 2016 -- December 2016}{Software Engineer Intern}{}
    \item Worked on the development of a semantic segmentation system for self-driving cars with poor computational resource.
    \end{rSubsection}

    \begin{rSubsection}{\href{https://www.fixstars.com/en/}{Fixstars Corporation}}{August 2014 -- September 2014}{Software Engineer Intern}{}
    \item Worked on software optimisation for the microcomputer of cars.
    \end{rSubsection}

\end{rSection}

%----------------------------------------------------------------------------------------
%	EXAMPLE SECTION
%----------------------------------------------------------------------------------------

\begin{rSection}{Publications} \itemsep 4pt

    \begin{rSubsection}{JOURNALS}{}{}{}{}
        \item \textbf{Naoya Muramatsu}, Hai-Tao Yu, Tetsuji Satoh, ``Combining Spiking Neural Networks with Artificial Neural Networks for Enhanced Image Classification,'' \textit{IEICE Transactions on Information and Systems}, 2023. doi: \url{10.1587/transinf.2021EDP7237}
    \end{rSubsection}

    \begin{rSubsection}{REFEREED CONFERENCES}{}{}{}{}
        \item \textbf{Naoya Muramatsu}*, Zico da Silva*, Daniel Joska, Fred Nicolls, Amir Patel, ``Improving 3D Markerless Pose Estimation of Animals in the Wild using Low-Cost Cameras,'' in Proc. IEEE/RSJ International Conference on Intelligent Robots and Systems (IROS 2022) (*co-first authors).
        \item Daniel Joska, Liam Clark, \textbf{Naoya Muramatsu}, Ricky Jericevich, Fred Nicolls, Alexander Mathis, Mackenzie Mathis, Amir Patel, ``AcinoSet: A 3D Pose Estimation Dataset and Baseline Models for Cheetahs in the Wild,'' in Proc. IEEE International Conference on Robotics and Automation (ICRA 2021).
        \item Chun Wei Ooi, \textbf{Naoya Muramatsu}, Yoichi Ochiai, ``Eholo glass: Electroholography glass. A lensless approach to holographic augmented reality near-eye display,'' in Technical Briefs of 11th ACM SIGGRAPH Asia (SA 2018).
        \item Natsumi Kato*, Hiroyuki Osone*, Daitetsu Sato, \textbf{Naoya Muramatsu}, Yoichi Ochiai, ``DeepWear: a Case Study of Collaborative Design between Human and Artificial Intelligence,''  in Proc. 12th ACM Twelfth International Conference on Tangible, Embedded, and Embodied Interaction (TEI 2018). (* co-first authors)
        \item Natsumi Kato, Hiroyuki Osone, Daitetsu Sato, \textbf{Naoya Muramatsu}, Yoichi Ochiai, ``Crowd Sourcing Clothes Design Directed by Adversarial Neural Networks,'' in  Adjunct Proc. 31st Neural Information Processing Systems (NIPS 2017 Workshop).
        \item \textbf{Naoya Muramatsu}, Chun Wei Ooi, Yuta Itoh, Yoichi Ochiai, ``DeepHolo: Recognizing 3D Objects using a Binary-weighted Computer-Generated Hologram,'' in Technical Briefs of 10th ACM SIGGRAPH Asia (SA 2017).
        \item Mose Sakashita, Yuta Sato, Ayaka Ebisu, Keisuke Kawahara, Satoshi Hashizume, \textbf{Naoya Muramatsu}, Yoichi Ochiai, ``Haptic Marionette: Wrist Control Technology Combined with Electrical Muscle Stimulation and Hanger Reflex,'' in in Adjunct Proc. 10th ACM SIGGRAPH Asia (SA 2017 Posters).
        \item \textbf{Naoya Muramatsu}, Ooi Chun Wei, Takashi Miyazaki, ``Development of High Performance Filter for Indoor Positioning System,'' in Proc. 5th IIAE International Conference on Intelligent Systems and Image Processing 2017 (ICISIP 2017).
        \item \textbf{Naoya Muramatsu}, Kazuki Ohshima, Ryota Kawamura, Ooi Chun Wei, Yuta Sato, Yoichi Ochiai, ``Sonoliards: Rendering Audible Sound Spots by Reflecting the Ultrasound Beams,'' in Adjunct Proc. 30th ACM User Interface Software and Technology (UIST 2017 Adjunct).
    \end{rSubsection}

    % \begin{rSubsection}{PREPRINTS}{}{}{}{}
    %     % arxivとか
    % \end{rSubsection}

    % \begin{rSubsection}{INTERNATIONAL CONFERENCES (NOT REFEREED)}{}{}{}{}
    % \end{rSubsection}

    % \begin{rSubsection}{DOMESTIC CONFERENCES (REFEREED)}{}{}{}{}
    % \end{rSubsection}

    \begin{rSubsection}{NON-REFEREED CONFERENCES}{}{}{}{}
        \item \textbf{Naoya Muramatsu}, Hai-Tao Yu, ``Combining Spiking Neural Network and Artificial Neural Network for Enhanced Image Classification,'' in Proc. 13th Data Engineering and Information Management (DEIM 2021).
        \item \textbf{Naoya Muramatsu}, Tetsuji Satoh, Takayasu Fushimi, ``Product Attribute Extraction Method Based on Transition Pattern of Review Point of View,'' in Proc. 9th Data Engineering and Information Management (DEIM 2017). (Student Presentation Award) % レビュー観点の推移パターンに基づく商品属性の抽出手法
    \end{rSubsection}

\end{rSection}

\begin{rSection}{Professional Experience} \itemsep 4pt
    \begin{rSubsection}{PAPER REVIEWER}{}{}{}{}
        \item IEEE/RSJ International Conference on Intelligent Robots and Systems (IROS), 2022.
        \item ACM Augmented Human International Conference (AH), 2018.
    \end{rSubsection}
    \begin{rSubsection}{CERTIFICATES}{}{}{}{}
        \item Operations Research (1): Models and Applications (Coursera), National Taiwan University, 2022.
        \item Kinematics: Describing the Motions of Spacecraft (Coursera), University of Colorado Boulder, 2022.
        \item Motion Planning for Self-Driving Cars (Coursera), University of Toronto, 2021.
        \item Robotics: Mobility (Coursera), University of Pennsylvania, 2021.
        \item Julia Scientific Programming (Coursera), University of Cape Town, 2021.
        \item State Estimation and Localization for Self-Driving Cars (Coursera), University of Toronto, 2021.
        \item Visual Perception for Self-Driving Cars (Coursera), University of Toronto, 2021.
    \end{rSubsection}
\end{rSection}

%----------------------------------------------------------------------------------------
\begin{rSection}{Fellowships}
    \begin{tabular}{ @{} >{\bfseries}l @{\hspace{4ex}} l }
        2023    &   Incoming International Student Scholarship at University of Cape Town (35,000 ZAR) \\
                &   Electrical Engineering P/G Scholarship at University of Cape Town (6,000 ZAR) \\
        2022    &   \href{https://www.microsoft.com/en-us/research/academic-program/phd-fellowship/africa/}{Microsoft Research PhD Fellowship Africa} (15,000 USD) \\
                &   Incoming International Student Scholarship at University of Cape Town (35,000 ZAR) \\
                &   Electrical Engineering P/G Scholarship at University of Cape Town (47,000 ZAR) \\
    \end{tabular}
\end{rSection}

%----------------------------------------------------------------------------------------
% \begin{rSection}{Research Grants}
%     \begin{tabular}{ @{} >{\bfseries}l @{\hspace{4ex}} l }
%     \end{tabular}
% \end{rSection}

%----------------------------------------------------------------------------------------
\begin{rSection}{Awards}
    \begin{tabular}{ @{} >{\bfseries}l @{\hspace{4ex}} l }
    2019    &   \textbf{Super Creator} at MITOU Projects  \\
            &   (This award were given 16 creators from 27 people adopted from 300+ applications) \\
    2018    &   MITOU Projects (2,304,000 JPY) \\
    2018    &   \textbf{President's Award for Students} at University of Tsukuba  \\
    2017    &   \textbf{Student Presentation Award} at DEIM 2017   \\
    2015    &   \textbf{Third Prize} at RoboCupJunior Soccer 2015 (Hokushinetsu Block)  \\
    \end{tabular}
\end{rSection}

%----------------------------------------------------------------------------------------
%	TECHNICAL STRENGTHS SECTION
%----------------------------------------------------------------------------------------

\begin{rSection}{Technical Strengths}

    \begin{tabular}{ @{} >{\bfseries}l @{\hspace{6ex}} l }
    Programming Languages   & Python(most fluent), MATLAB, Julia, C, C++, Verilog, \\
                            & Shell Script, SQL \\
    Software    &   PyTorch, Tensorflow, OpenCV, ROS, Pyomo, \\
                &   Docker, PyBullet, IPOPT, Processing \\
    OS          &   MacOS, Ubuntu, Windows, TrueNAS, CentOS \\
    Hardware    &   mbed, Arduino, Raspberry Pi, Hexapod \\
    \end{tabular}

\end{rSection}


%----------------------------------------------------------------------------------------
\begin{rSection}{Links}
    {\bf Portfolio} \hfill {\url{https://denden047.github.io/index_en.html}} \\
    {\bf GitHub} \hfill {\url{https://github.com/DenDen047}} \\
    {\bf Linkedin} \hfill {\url{https://www.linkedin.com/in/naoya-muramatsu-a01182184/}} \\
\end{rSection}

\end{document}
