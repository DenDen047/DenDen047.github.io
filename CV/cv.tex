%%%%%%%%%%%%%%%%%%%%%%%%%%%%%%%%%%%%%%%%%
% Medium Length Professional CV
% LaTeX Template
% Version 2.0 (8/5/13)
%
% This template has been downloaded from:
% http://www.LaTeXTemplates.com
%
% Original author:
% Trey Hunner (http://www.treyhunner.com/)
%
% Important note:
% This template requires the resume.cls file to be in the same directory as the
% .tex file. The resume.cls file provides the resume style used for structuring the
% document.
%
%%%%%%%%%%%%%%%%%%%%%%%%%%%%%%%%%%%%%%%%%

%----------------------------------------------------------------------------------------
%	PACKAGES AND OTHER DOCUMENT CONFIGURATIONS
%----------------------------------------------------------------------------------------

\documentclass{resume} % Use the custom resume.cls style

% コメント埋め込み用マクロ
\usepackage{color}
\usepackage{url}
\newcommand{\todo}[1]{\textsf{\textcolor{red}{\textbf{TODO:} \textit{#1}}}}


% データセット名マクロ
\usepackage{xspace}

% 省略形
\newcommand{\etal}{\textit{et al}.}
\newcommand{\ie}{\textit{i}.\textit{e}.}
\newcommand{\eg}{\textit{e}.\textit{g}.}
	% マクロを読み込む

\usepackage[left=0.75in,top=0.6in,right=0.75in,bottom=0.6in]{geometry} % Document margins
\usepackage[dvipdfmx]{hyperref}
\newcommand{\tab}[1]{\hspace{.2667\textwidth}\rlap{#1}}
\newcommand{\itab}[1]{\hspace{0em}\rlap{#1}}
\name{Naoya Muramatsu}	% Your name
% \address{\href{http://www.u.tsukuba.ac.jp/~s1411453/}{http://www.u.tsukuba.ac.jp/~s1411453/}}
\address{sh.mn.nat@gmail.com}	% Your phone number and email
\address{Yashio, Saitama Pref., Japan}	% Your address
% \address{123 Pleasant Lane \\ City, State 12345} % Your secondary address (optional)


\begin{document}

%----------------------------------------------------------------------------------------
%	EDUCATION SECTION
%----------------------------------------------------------------------------------------

\begin{rSection}{Education}

    {\bf University of Tsukuba} \hfill {April 2018 -- Present}
    \\ Master of Information Science
    \\ Department of Library, Information and Media Studies, Graduate School of Library, Information and Media Studies
    \\ Adviser: Tetsuji Satoh
    \\ Subadviser: Yu Hai-Tao

    {\bf University of Tsukuba} \hfill {April 2016 -- March 2018}
    \\ Bachelor of Library and Information Science
    \\ College of Knowledge and Library Sciences
    \\ Adviser: Yoichi Ochiai
    % \\ GPA: \todo{xxx / 4.03}

    {\bf National Institute of Technology, Nagano College} \hfill {April 2011 -- March 2016}
    \\ Foundation Degree
    \\ Department of Electrical and Electronic Engineering
    \\ Adviser: Takashi Miyazaki
    % \\ GPA: \todo{xxx / 4.03}

\end{rSection}

%----------------------------------------------------------------------------------------
%	WORK EXPERIENCE SECTION
%----------------------------------------------------------------------------------------

\begin{rSection}{Research Experience}

    \begin{rSubsection}{University of Tsukuba}{April 2018 -- Present}{Master Research}{}
    % New type of deep learning
    \item Proposed the method of applying spiking neural networks for event-based cameras.
    \end{rSubsection}

    \begin{rSubsection}{University of Cape Town}{July 2020}{Research Internship}{}
    % New type of deep learning
    \item Estimated the motion of a real cheetah from multi-cameras with trajectory optimization (Supervisor: \href{https://scholar.google.co.za/citations?user=RxMigV4AAAAJ&view_op=list_works&sortby=pubdate}{Dr. Amir Patel}).
    \end{rSubsection}

    \begin{rSubsection}{University of Tsukuba}{April 2016 -- March 2018}{Undergraduate Research}{}
    % ECサイトのレビュー
    \item Analyzed reviews of EC site to find out points of variation.
    % Sonoliards
    \item Developed Sonoliards optimizing the direction of a parametric speaker with a ray tracing algorithm.
    % DeepHolo
    \item Developed DeepHolo that recognizes 3D objects using a deep neural network and computer-generated holography for convert 3D data to 2D data with depth information.
    % DeepWear
    \item Developed DeepWear, a method using deep convolutional generative adversarial networks (DCGANs) for clothes design.
    \end{rSubsection}

    \begin{rSubsection}{National Institute of Technology, Nagano College}{April 2015 -- March 2016}{Undergraduate Research}{}
    \item Developed the noise filter that greatly suppresses the influence of radio noise in the indoor position information system using Link Quality Indication (LQI) value of radio waves.
    \end{rSubsection}

\end{rSection}

%----------------------------------------------------------------------------------------
%	TECHNICAL STRENGTHS SECTION
%----------------------------------------------------------------------------------------

\begin{rSection}{Technical Strengths}

    \begin{tabular}{ @{} >{\bfseries}l @{\hspace{6ex}} l }
    Programming Languages   &   Python(most fluent), C, C++, Verilog, Shell Script, \\
    & Ruby, JavaScript, SQL \\
    Machine Learning Libraries & Tensorflow, Keras, PyTorch, Scikit-learn, Chainer \\
    Software    &   Git, Docker, PyBullet, Processing, Autodesk Fusion360 \\
    OS          &   MacOS, Ubuntu, Windows, FreeNAS, CentOS \\
    Hardware    &   Arduino, Mbed, PhantomX AX Metal Hexapod \\
    \end{tabular}

\end{rSection}

%----------------------------------------------------------------------------------------
%	EXAMPLE SECTION
%----------------------------------------------------------------------------------------

\begin{rSection}{Publications} \itemsep 4pt

    % \begin{rSubsection}{Preprints}{}{}{}{}
    %     % arxivとか
    % \end{rSubsection}

    % \begin{rSubsection}{JOURNAL ARTICLES}{}{}{}{}
    % \end{rSubsection}

    \begin{rSubsection}{INTERNATIONAL CONFERENCES (REFEREED)}{}{}{}{}
        \item Liam Clark, \textbf{Naoya Muramatsu}, Daniel Joska, Ricky Jericevich, Fred Nicolls, Alexander Mathis, Mackenzie Mathis, Amir Patel. 2021. AcinoNet: 3D Markerless Motion Tracking of Cheetahs in the Wild. In \textit{International Conference on Robotics and Automation} (ICRA 2021). IEEE, Xi' an, China.

        \item Chun Wei Ooi, \textbf{Naoya Muramatsu}, and Yoichi Ochiai. 2018. Eholo glass: Electroholography glass. A lensless approach to holographic augmented reality near-eye display. In \textit{SIGGRAPH Asia 2018 Technical Briefs} (SA ’18), December 4–7, 2018, Tokyo, Japan. ACM, New York, NY, USA, 4 pages. DOI: \url{https://doi.org/10.1145/3283254.3283288}

        \item Natsumi Kato*, Hiroyuki Osone*, Daitetsu Sato, \textbf{Naoya Muramatsu}, and Yoichi Ochiai. 2018. DeepWear: a Case Study of Collaborative Design between Human and Artificial Intelligence. In \textit{Proceedings of the Twelfth International Conference on Tangible, Embedded, and Embodied Interaction} (TEI ’18). ACM, New York, NY, USA, 529-536. DOI: \url{https://doi.org/10.1145/3173225.3173302} (* Joint first authorship.)

        \item \textbf{Naoya Muramatsu}, Ooi Chun Wei, Takashi Miyazaki. 2017. Development of High Performance Filter for Indoor Positioning System. In \textit{The 5th IIAE International Conference on Intelligent Systems and Image Processing 2017}(ICISIP 2017).
    \end{rSubsection}

    \begin{rSubsection}{INTERNATIONAL POSTERS AND WORKSHOPS (REFEREED)}{}{}{}{}
        \item Natsumi Kato, Hiroyuki Osone, Daitetsu Sato, \textbf{Naoya Muramatsu}, and Yoichi Ochiai. 2017. Crowd Sourcing Clothes Design Directed by Adversarial Neural Networks. In \textit{NIPS 2017 Workshop} (NIPS ’17).

        \item \textbf{Naoya Muramatsu}, Kazuki Ohshima, Ryota Kawamura, Ooi Chun Wei, Yuta Sato, and Yoichi Ochiai. 2017. Sonoliards: Rendering Audible Sound Spots by Reflecting the Ultrasound Beams. In \textit{Adjunct Publication of the 30th Annual ACM Symposium on User Interface Software and Technology} (UIST ’17). ACM, New York, NY, USA, 57-59. DOI: \url{https://doi.org/10.1145/3131785.3131807}

        \item \textbf{Naoya Muramatsu}, Chun Wei Ooi, Yuta Itoh, and Yoichi Ochiai. 2017. DeepHolo: Recognizing 3D Objects using a Binary-weighted Computer-Generated Hologram. In \textit{SIGGRAPH Asia 2017 Posters} (SA 2017), November 27– 30, 2017, Bangkok, Thailand. ACM, New York, NY, USA, 2 pages. DOI: \url{https://doi.org/10.1145/3145690.3145725}

        \item Mose Sakashita, Yuta Sato, Ayaka Ebisu, Keisuke Kawahara, Satoshi Hashizume, \textbf{Naoya Muramatsu}, Yoichi Ochiai. 2017. Haptic Marionette: Wrist Control Technology Combined with Electrical Muscle Stimulation and Hanger Reflex. In \textit{SIGGRAPH Asia 2017 Posters} (SA 2017). ACM, New York, NY, USA, Article 33, 2 pages. DOI: \url{https://doi.org/10.1145/3145690.3145743}
    \end{rSubsection}

    % \begin{rSubsection}{INTERNATIONAL CONFERENCES (NOT REFEREED)}{}{}{}{}
    % \end{rSubsection}

    % \begin{rSubsection}{DOMESTIC CONFERENCES (REFEREED)}{}{}{}{}
    % \end{rSubsection}

    \begin{rSubsection}{DOMESTIC CONFERENCES (NOT REFEREED)}{}{}{}{}
        \item \textbf{Naoya Muramatsu}, Hai-Tao Yu. 2021. Combining Spiking Neural Network and Artificial Neural Network for Enhanced Image Classification. In \textit{Data Engineering and Information Management 2021} (DEIM 2021).
        \item \textbf{Naoya Muramatsu}, Tetsuji Satoh, Takayasu Fushimi. 2017. Product Attribute Extraction Method Based on Transition Pattern of Review Point of View. In \textit{Data Engineering and Information Management 2017} (DEIM 2017). (in Japanese) % レビュー観点の推移パターンに基づく商品属性の抽出手法
    \end{rSubsection}




\end{rSection}

%----------------------------------------------------------------------------------------
\begin{rSection}{Work Experience}

    \begin{rSubsection}{\href{https://www.landscape.co.jp/}{Landscape Co.,Ltd.}}{January 2020 -- Present}{Outside CTO}{}
    \item Worked on development of some systems with Machine Learning techniques.
    \end{rSubsection}

    \begin{rSubsection}{Information-technology Promotion Agency, Japan. Exploratory Software Project (\href{https://www.mitou.org/}{MITOU})}{June 2018 -- March 2019}{Creator}{}
    \item Developed the robot control system, able to walk even if a few legs are broken using hierarchy Q-learning.
    \item 2,304,000 JPY / nine months.
    \end{rSubsection}

    \begin{rSubsection}{\href{http://pixiedusttech.com/}{Pixie Dust Technologies, Inc.}}{August 2017 -- April 2019}{Software Engineer}{}
    \item Worked on development of management systems and web applications.
    \end{rSubsection}

    \begin{rSubsection}{\href{https://www.fixstars.com/en/}{Fixstars Corporation}}{August 2016 -- December 2016}{Software Engineer Intern}{}
    \item Worked on development of semantic segmentation system for self-driving cars.
    \end{rSubsection}

    \begin{rSubsection}{\href{https://www.fixstars.com/en/}{Fixstars Corporation}}{August 2014 -- September 2014}{Software Engineer Intern}{}
    \item Worked on software optimization for the microcomputer of cars.
    \end{rSubsection}

\end{rSection}

%----------------------------------------------------------------------------------------
% \begin{rSection}{Fellowships}
%     \begin{tabular}{ @{} >{\bfseries}l @{\hspace{4ex}} l }
%     \end{tabular}
% \end{rSection}

%----------------------------------------------------------------------------------------
% \begin{rSection}{Research Grants}
%     \begin{tabular}{ @{} >{\bfseries}l @{\hspace{4ex}} l }
%     \end{tabular}
% \end{rSection}

%----------------------------------------------------------------------------------------
\begin{rSection}{Awards}
    \begin{tabular}{ @{} >{\bfseries}l @{\hspace{4ex}} l }
    2018    &   MITOU Projects, \textbf{Super Creator}  \\
            &   This award were given 16 creators from 27 people adopted from 300+ applications. \\
    2018    &   University of Tsukuba, \textbf{President's Award for Students}  \\
    2017    &   DEIM 2017, \textbf{Student Presentation Award}.   \\
    2015    &   RoboCupJunior Soccer 2015 in Hokushinetsu Block, \textbf{Prize: 3rd} \\
    \end{tabular}
\end{rSection}


%----------------------------------------------------------------------------------------
\begin{rSection}{Links}

    {\bf Digital Nature Group} \hfill {\url{https://digitalnature.slis.tsukuba.ac.jp/}} \\
    {\bf GitHub: DenDen047} \hfill {\url{https://github.com/DenDen047}} \\

\end{rSection}

\end{document}
